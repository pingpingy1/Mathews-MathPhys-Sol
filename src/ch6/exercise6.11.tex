%%%%%%%%%%%%%%%%%%%%%%%
\item
%%%%%%%%%%%%%%%%%%%%%%%

The problem, as stated, is wrong; its proposed solution is for when
the cyclic boundary condition has been lifted.
Let us solve the problem backwards, incrementing the imposed conditions.

When there are no constraints, the system is $N$ independent atoms,
each with 3 allowed states.
Thus, $W = 3^N$ and $S = k \ln 3$.

When the AC condition is imposed, let $v_A^{(j)}, v_B^{(j)}, \text{ and } v_C^{(j)}$
denote the possible number of states with $j$ atoms such that the last atom is in state $A, B, \text{ and } C$, respectively.
Finding a recurrence relation for these is then a classic combinatorics problem.
Every $(j+1)$-atom state ending in an A tom can be bijectively mapped
to a $j$-atom state ending in a non-C atom.
That is, $v_A^{(j + 1)} = v_A^{(j)} + v_B^{(j)}$.
Similar arguments yield the transfer matrix given in the problem.
Therefore, we have
\begin{align*}
    W
    &= \begin{pmatrix}
        1 & 1 & 1
    \end{pmatrix} M^N \begin{pmatrix}
        v_A^{(1)} \\ v_B^{(1)} \\ v_C^{(1)}
    \end{pmatrix} \\
    &\approx \begin{pmatrix}
        1 & 1 & 1
    \end{pmatrix}
    \cdot {\left( 1 + \sqrt{2} \right)}^N \begin{pmatrix}
        \half \\ \inv{\sqrt{2}} \\ \half
    \end{pmatrix} \begin{pmatrix}
        \half & \inv{\sqrt{2}} & \half
    \end{pmatrix} \cdot \begin{pmatrix}
        1 \\ 1 \\ 1
    \end{pmatrix} \\
    &= {\left( 1 + \inv{\sqrt{2}} \right)}^2 {\left( 1 + \sqrt{2} \right)}^N
\end{align*}
where the vector $\begin{pmatrix} \half \\ \inv{\sqrt{2}} \\ \half \end{pmatrix}$ is the unit eigenvector
corresponding to the largest eigenvalue of $1 + \sqrt{2}$.
\[
    \therefore S
    = \lim_{N \rightarrow \infty} \frac{k \ln W}{N}
    = k \ln \left( 1 + \sqrt{2} \right)
\]

When the circular boundary condition is imposed,
we must define separate variables for the starting atom's state as well.
Hence, we define the 7-dimensional vector
\[
    \mathbb{v}^{(j)} := \begin{pmatrix}
        v^{(j)}_{AA} \\
        v^{(j)}_{AB} \\
        v^{(j)}_{BA} \\
        v^{(j)}_{BB} \\
        v^{(j)}_{BC} \\
        v^{(j)}_{CB} \\
        v^{(j)}_{CC}
    \end{pmatrix}.
\]
The transfer matrix is then defined by
\[
    M = \left(\begin{array}{cc|ccc|cc}
         1 & 1 &   &   &   &   &   \\
         1 & 1 &   &   &   &   &   \\
         \hline
           &   & 1 & 1 &   &   &   \\
           &   & 1 & 1 & 1 &   &   \\
           &   &   & 1 & 1 &   &   \\
        \hline
           &   &   &   &   & 1 & 1 \\
           &   &   &   &   & 1 & 1
    \end{array}\right).
\]
The dominating eigenvalue and the corresponding eigenvector
are equal to the previous case.
\begin{align*}
    \Rightarrow W
    &= \begin{pmatrix}
        1 & 1 & 1 & 1 & 1 & 1 & 1
    \end{pmatrix}
    M^N \begin{pmatrix}
        1 \\ 0 \\ 0 \\ 1 \\ 0 \\ 0 \\ 1
    \end{pmatrix} \\
    &\approx {\left( 1 + \sqrt{2} \right)}^N
    \begin{pmatrix}
        1 & 1 & 1 & 1 & 1 & 1 & 1
    \end{pmatrix}
    \begin{pmatrix}
        0 \\ 0 \\ \half \\ \inv{\sqrt{2}} \\ \half \\ 0 \\ 0
    \end{pmatrix} \\
    &\quad \times \begin{pmatrix}
        0 & 0 & \half & \inv{\sqrt{2}} & \half & 0 & 0
    \end{pmatrix} \begin{pmatrix}
        1 \\ 0 \\ 0 \\ 1 \\ 0 \\ 0 \\ 1
    \end{pmatrix} \\
    &= \inv{\sqrt{2}} \left( 1 + \inv{\sqrt{2}} \right) {\left( 1 + \sqrt{2} \right)}^N
\end{align*}
The entropy is therefore the same as before.
