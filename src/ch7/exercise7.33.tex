%%%%%%%%%%%%%%%%%%%%%%%
33.
%%%%%%%%%%%%%%%%%%%%%%%
The simple nature of $f(x)$ prompts us not to blindly make a Fourier expansion.
Indeed, $y$ is linear when $f(x) = 0$ and sinusoidal when $f(x) = C$
with angular frequency $\sqrt{C}$.
One possible method is to define the general solution
\[
    y(x) = \begin{cases}
        A_n x + B_n & (2n\pi < x < (2n + 1) \pi) \\
        D_n \cos \sqrt{C}x + E_n \sin \sqrt{C}x & ((2n + 1) \pi < x < 2(n + 1)\pi)
    \end{cases}
\]
and find recursion formulae for the coefficients.
However, after days of making algebraic mistakes and starting over and over again,
we may take a simpler route.

Suppose $y_1(x)$ and $y_2(x)$ are two independent solutions.
As noted in the textbook, we may write
\[
    \begin{pmatrix}
        y_1(x + 2\pi) \\ y_2(x + 2\pi)
    \end{pmatrix}
    = \begin{pmatrix}
        A_{11} & A_{12} \\
        A_{21} & A_{22}
    \end{pmatrix} \begin{pmatrix}
        y_1(x) \\ y_2(x)
    \end{pmatrix}
\]
for some coefficients $A_{ij} (i, j = 1, 2)$.
Then, $e^{2\pi\mu}$ is an eigenvalue of the coefficient matrix.
That is, we only need to find the coefficient matrix
even if $y_1$ and $y_2$ are not exactly found.

As such, we arbitrarily assume $y_1(x) = 1$ and $y_2(x) = x$ for $0 < x < \pi$.
Using the continuity of $y$ and $y'$, we extend these solutions up to $2\pi < x < 3\pi$
and extract the coefficient matrix.

\[
    y_1(x) = \begin{cases}
        1 & (0 < x < \pi) \\
        \cos\left( \sqrt{C} (x - \pi) \right) & (\pi < x < 2\pi) \\
        -\sqrt{C}\sin\left( \pi\sqrt{C} \right)
        \left( x - 2\pi \right) + \cos \left( \pi \sqrt{C} \right)
        & (2\pi < x < 3\pi)
    \end{cases}
\]
\[
    y_2(x) = \begin{cases}
        x - \pi & (0 < x < \pi) \\
        \inv{\sqrt{C}} \sin\left( \sqrt{C}(x - \pi) \right) & ( \pi < x < 2\pi) \\
        \cos\left( \pi\sqrt{C} \right) (x - 2\pi) + \inv{\sqrt{C}}\sin\left( \pi\sqrt{C} \right)
        & (2\pi < x < 3\pi)
    \end{cases}
\]
\[
    \Rightarrow \left( A_{ij} \right)
    = \begin{pmatrix}
        -\sqrt{C}\sin\left( \pi\sqrt{C} \right) + \cos\left( \pi\sqrt{C} \right)
        & - \sqrt{C} \sin\left( \pi \sqrt{C} \right)\\
        \pi\cos\left(\pi\sqrt{C} \right) + \inv{\sqrt{C}} \sin\left( \pi\sqrt{C} \right)
        & \cos\left(\pi\sqrt{C}\right)
    \end{pmatrix}
\]
\begin{align*}
    \therefore \mu
    &= \inv{2\pi} \ln\left(
        \cos\left( \pi\sqrt{C} \right)
        - \frac{\pi\sqrt{C}}{2} \sin\left( \pi\sqrt{C} \right)
    \right. \\
    &\quad \left.
        \pm \sqrt{
            -\pi^2\sqrt{C} \cos^2\left( \pi\sqrt{C} \right)
            -\pi\sin\left( \pi\sqrt{C} \right) \cos\left( \pi\sqrt{C} \right)
            + \frac{\pi^2C}{4}\sin^2\left( \pi\sqrt{C} \right)
        }
    \right)
\end{align*}
A solution with period $2\pi$ is possible when either one of $\mu$ equals 0;
a periodic solution exists when either one is a rational multiple of $2i\pi$.

