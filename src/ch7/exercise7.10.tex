%%%%%%%%%%%%%%%%%%%%%%%
10.
%%%%%%%%%%%%%%%%%%%%%%%
Let us approach the problem using a general second-order linear ODE.
That is, suppose $f$ and $g$ are solutions to the differential equation
\[
    p(x) \frac{d^2 y}{dx^2} + q(x) \frac{dy}{dx} + r(x) = 0.
\]
Define the Wronskian as
\[
    W(x) := f(x) g'(x) - f'(x) g(x).
\]
We then have
\begin{align*}
    p(x) W'(x)
    &= p(x) \left(\
        f'(x) g'(x) + f(x) g''(x) - f''(x) g(x) - f'(x) g'(x)
    \right)\\
    &= f(x) \left( p(x) g''(x) \right)
     - \left( p(x) f''(x) \right) g(x) \\
    &= f(x) \left( -q(x) g'(x) - r(x) g(x) \right)
     - \left( -q(x) f'(x) - r(x) f(x) \right) g(x) \\
    &= -q(x) W(x).
\end{align*}
Therefore, the Wronskian follows the first-order linear ODE
\[
    p(x) \frac{dW}{dx}(x) + q(x) W(x) = 0.
\]

In the case of Bessel's equation, $p(x) = x^2$ and $q(x) = x$.
\[
    \Rightarrow W'(x) = -\inv{x} W(x)
\]
\[
    \therefore W(x) = \frac{C}{x}
\]

