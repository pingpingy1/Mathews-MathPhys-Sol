%%%%%%%%%%%%%%%%%%%%%%%
\item
%%%%%%%%%%%%%%%%%%%%%%%

Parallelling the textbook, suppose the differential equation is
\[
    y'' + \frac{\textrm{poly}_n(z)}{(z - \xi)(x - \eta)} y'
    + \frac{\textrm{poly}_m (z)}{{(z - \xi)}^2 {(z - \eta)}^2} y = 0
\]
where $n$ and $m$ are the degrees of the respective polynomials.
The variable change $u = \inv{z}$ yields
\[
    \frac{d^2 y}{du^2}
    + \left(
        \frac{2}{u} - \frac{
            \textrm{poly}_n \left( u^{-1} \right)
        }{(1 - \xi u) (1 - \eta u)}
    \right) \frac{dy}{du}
    + \frac{\textrm{poly}_m \left( u^{-1} \right)}{{(1 - \xi u)}^2 {(1 - \eta u)}^2} y
    = 0.
\]
For $z = \infty \Leftrightarrow u = 0$ to be ordinary, we must have $n = 1$ and $m = 0$.
\[
    \Rightarrow y''
    + \left( \frac{A}{z - \xi} + \frac{B}{z - \eta} \right) y'
    + \frac{C}{{(z - \xi)}^2 {(z - \eta)}^2} y
    = 0
\]
\[
    \frac{d^2 y}{du^2}
    + \inv{u} \left(
        2 - \frac{A}{1 - \xi u} - \frac{B}{1 - \eta u}
    \right) \frac{dy}{du}
    + \frac{C}{{(1 - \xi u)}^2 {(1 - \eta u)}^2} y
    = 0
\]
$u = 0$ being an ordinary point further demands that $A + B = 2$.

Now, suppose $y \sim {(z - \xi)}^\alpha$ near $z = \xi$.
\[
    0
    = \alpha (\alpha - 1) + A \alpha + \frac{C}{{(\xi - \eta)}^2}
    = \alpha^2 - (A - 1) \alpha + \frac{C}{{(\xi - \eta)}^2}
\]
\[
    \Rightarrow \alpha_1 + \alpha_2 = A - 1,\;
    \alpha_1 \alpha_2 = \frac{C}{{(\xi - \eta)}^2}
\]
Similar analysis for $y \sim {(z - \eta)}^\beta$ near $z = \eta$ yields
\[
    \beta_1 + \beta_2 = 1 - B,\;
    \beta_1 \beta_2 = \frac{C}{{(\xi - \eta)}^2}.
\]
Therefore, eliminating $A,\; B,$ and $C$, we get the most general form
\[
    y'' + \left(
        \frac{1 - \alpha_1 - \alpha_2}{z - \xi}
        + \frac{1 - \beta_1 - \beta_2}{z - \eta}
    \right) y' + \frac{{(\xi - \eta)}^2 \alpha_1 \alpha_2}{
        {(z - \xi)}^2 {(z - \eta)}^2
    } y = 0 \quad
\]
with conditions
\[
    \alpha_1 + \alpha_2 + \beta_1 + \beta_2 = 0,\;
    \alpha_1 \alpha_2 = \beta_1 \beta_2.
\]

The given transformation of exponentials leaves the sum of them fixed.
As for the product,
\begin{align*}
    (\alpha_1 + \lambda) (\alpha_2 + \lambda)
    &= \alpha_1 \alpha_2 + \lambda (\alpha_1 + \alpha_2) + \lambda^2 \\
    &= \beta_1 \beta_2 - \lambda (\beta_1 + \beta_2) + \lambda^2 \\
    &= (\beta_1 - \lambda) (\beta_2 - \lambda).
\end{align*}
Therefore, the given transformation leaves the condition satisfied.
