%%%%%%%%%%%%%%%%%%%%%%%
\item
%%%%%%%%%%%%%%%%%%%%%%%

\begin{figure}[ht]
	\centering
	\begin{tikzpicture}
		% Axes
		\draw[axis] (-3.5cm, 0cm) -- (1cm, 0cm) node[right] {$\Re$};
		\draw[axis] (0cm, -3.5cm) -- (0cm, 1cm) node[right] {$\Im$};

        % Branch cut
		\node[pole] at (0cm, 0cm) {};
		\draw[branch cut] (-3.4cm, -3.4cm) -- (0cm, 0cm);


        % \draw[contour] plot [smooth] coordinates {
        %     (0cm, 1cm) (-0.5cm, 1.2cm) (-2cm, 0.5cm) (-3.9cm, 0.1cm)
        % }; % Lol this looks horrible

		% Contour
        \draw[contour] (0cm, 0cm) arc (0:90:0.707cm) node[above] {$\half H_n^{(1)}$};
        \draw[contour] plot [smooth] coordinates {
            (-0.707cm, 0.707cm) (-1.2cm, 0.4cm) (-2cm, -1.5cm) (-3.5cm, -3.4cm)
        };

        \draw[contour, magenta] (0cm, 0cm) arc (90:0:0.707cm) node[right] {$\half H_n^{(2)}$};
        \draw[contour, magenta] plot [smooth] coordinates {
            (0.707cm, -0.707cm) (0.4cm, -1.2cm) (-1.5cm, -2cm) (-3.4cm, -3.5cm)
        };

	\end{tikzpicture}
	\caption{Contour for \refprob{}~7--19.}%
	\label{fig:problem7-19}
\end{figure}

Notice that the region of validity for a specific contour for the Hankel functions
is the half-plane in $\mathbb{C}$ perpendicular to the branch cut.
Thus, we make the cut along $\arg z = - \frac{3\pi}{4}$, as shown in \reffig~\ref{fig:problem7-19}.