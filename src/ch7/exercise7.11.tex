%%%%%%%%%%%%%%%%%%%%%%%
\item
%%%%%%%%%%%%%%%%%%%%%%%

As shown in the previous exercise, we now know that
\[
    W(x) := J_m(x) J_{-m}'(x) - J_m'(x) J_{-m}(x) = \frac{C_m}{x}
\]
for some constant $C_m$.
As this holds for all $x$, we may write
\begin{align*}
    C_m
    &= \lim_{x \rightarrow 0} x \left(
        J_m(x) J_{-m}'(x) - J_m'(x) J_{-m}(x)
    \right) \\
    &= \lim_{x \rightarrow 0} \frac{x}{2} \left(
        J_m \left( J_{-m - 1} - J_{-m + 1} \right)
        - \left( J_{m - 1} - J_{m + 1} \right) J_{-m}
    \right).
\end{align*}
Notice that as $x \rightarrow 0$, each term behaves like $J_m \sim x^m$.
This argument shows that the terms like $x J_{m} J_{-m-1}$ vanish at the origin.
\begin{align*}
    \Rightarrow C_m
    &= \lim_{x \rightarrow 0} \frac{x}{2} \left(
        J_m (x) J_{-m - 1}(x) - J_{m - 1}(x) J_{-m} (x)
    \right) \\
    &= \lim_{x \rightarrow 0} \frac{x}{2} \left(
        \frac{x^m}{2^m m!} \frac{x^{-m-1}}{2^{-m-1} (-m-1)!}
        - \frac{x^{m-1}}{2^{m-1} (m-1)!} \frac{x^{-m}}{2^{-m} (-m)!}
    \right) \\
    &= \inv{\Gamma(m + 1)\Gamma(-m)} - \inv{\Gamma(m)\Gamma(-m+1)} \\
    &= \inv{\Gamma(m)\Gamma(-m)} \left( \inv{-m} - \inv{-m} \right) \\
    &= -\frac{2}{\Gamma(m)\Gamma(1-m)} \\
    &= -\frac{2}{\pi} \sin m\pi
\end{align*}
\[
    \therefore W(x) = -\frac{2\sin m\pi}{\pi x}
\]
We can verify that $W(x) = 0$ for integral values of $m$.
