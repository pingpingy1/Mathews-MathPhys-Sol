%%%%%%%%%%%%%%%%%%%%%%%
14.
%%%%%%%%%%%%%%%%%%%%%%%
\begin{figure}[ht]
	\centering
	\begin{tikzpicture}
		% Axes
		\draw[axis] (-4cm, 0cm) -- (0.5cm, 0cm) node[right] {$\Re$};
		\draw[axis] (0cm, -2cm) -- (0cm, 2cm) node[right] {$\Im$};

        % Branch cut
		\node[pole] at (0cm, 0cm) {};
		\draw[branch cut] (-3.9cm, 0cm) -- (0cm, 0cm);

		% Contour
        \draw[contour] (0cm, 0cm) arc (-45:45:0.707cm);
        \draw[contour] plot [smooth] coordinates {
            (0cm, 1cm) (-0.5cm, 1.2cm) (-2cm, 0.5cm) (-3.9cm, 0.1cm)
        }; % Lol this looks horrible
        \node[right] at (0cm, 1cm) {$i$};

	\end{tikzpicture}
	\caption{Contour for \refprob{}~7--14.}%
	\label{fig:problem7-14}
\end{figure}

The integral formulation for the Hankel function is given by
\[
    H_n^{(1)}(z) = \inv{i\pi} \int_C dt \frac{
        \exp\left( \frac{z}{2} \left( t - \inv{t} \right) \right)
    }{t^{n + 1}}
\]
along the contour shown in \reffig~\ref{fig:problem7-14}
The numerator has a saddle point at $t = i$ along the contour.
Around this point,
\[
    t + \inv{t} \approx 2i - e^{i\halfpi} {(t - i)}^2.
\]
\[
    \Rightarrow H_n^{(1)}(z)
    \approx \inv{i\pi} \int dt \frac{
        iz - \frac{z}{2} e^{i\pi} {(t - i)}^2
    }{t^{n + 1}}
\]
By what angle should we cross the $t = i$ point to perform the saddle-point method?
Since the coefficient of the quadratic term has argument $\pi$,
the contour must pass the point with angle $\frac{3\pi}{4}$ or $\frac{5\pi}{4}$.
The shape of the contour demands $\phi = \frac{3\pi}{4}$.

\begin{align*}
    \therefore H_n^{(1)}(z)
    &\approx \frac{e^{i\left( \frac{3\pi}{4} + z \right)}}{i^{n + 2} \pi}
     \cdot \int_{-\infty}^\infty dr \exp\left( -\frac{z}{2} r^2 \right) \\
    &= \sqrt{\frac{2}{\pi z}} \exp\left( iz - i \frac{2n + 1}{4}\pi \right)
\end{align*}

