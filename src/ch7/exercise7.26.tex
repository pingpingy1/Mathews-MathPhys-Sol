%%%%%%%%%%%%%%%%%%%%%%%
26.
%%%%%%%%%%%%%%%%%%%%%%%
\[
    y'' + \left( \inv{z} + \inv{z - 1} \right) y'
    + \inv{z (z - 1) (z - 2)} \left( \frac{-\frac{2}{9}}{z} \right) y
    = 0
\]
This equation is the \urlfoot{%
    https://en.wikipedia.org/wiki/Riemann\%27s_differential_equation
}{hypergeometric equation}\footnote{%
Also known as \refeq{7}{80} of the textbook, Riemann's equation, or Papperitz equation.}
with the regular singularities at $z = 0, 1, 2$ and corresponding powers
\[
    \alpha_1 = -\inv{3},\; \alpha_2 = \inv{3},\;
    \beta_1 = \beta_2 = \gamma_1 = 0,\; \gamma_2 = 1.
\]
\[
    \Rightarrow y
    = \RiemannP{
        \threemat{0}{1}{2}{-\inv{3}}{0}{0}{\inv{3}}{0}{1}
    }{z}
    = {\left( \frac{z - 2}{z} \right)}^{\inv{3}} \RiemannP{
        \threemat{0}{1}{2}{0}{0}{-\inv{3}}{\frac{2}{3}}{0}{\frac{2}{3}}
    }{z}
\]
We wish to transform $z$ such that 0 and 1 remain fixed, while 2 transforms to infinity.
This is achieved by the homographic transformation $u = \frac{z}{2 - z}$.
\[
    \Rightarrow y
    = {(-u)}^{-\inv{3}} \RiemannP{
        \threemat{0}{1}{\infty}{0}{0}{-\inv{3}}{\frac{2}{3}}{0}{\frac{2}{3}}
    }{u}
\]
This is the P symbol corresponding to the hypergeometric equation
with $a = -\inv{3},\; b = \frac{2}{3},$ and $c = \inv{3}$.
Therefore, substituting $u$ back in, we obtain the final solution
\begin{align*}
    y
    &= C_1 {\left( \frac{z}{2 - z} \right)}^{\inv{3}} \hypergeom \left(
        -\inv{3}, \frac{2}{3}; \inv{3}; \frac{z}{2 - z}
    \right) \\
    &\quad + C_2 {\left( \frac{2 - z}{z} \right)}^{\inv{3}} \hypergeom \left(
        \inv{3}, \frac{4}{3}; \frac{5}{3}; \frac{z}{2 - z}
    \right).
\end{align*}

