%%%%%%%%%%%%%%%%%%%%%%%
11.
%%%%%%%%%%%%%%%%%%%%%%%
The crucial insight to solve this problem (at least in my case) was the realization that
the transformation $f(x) \mapsto g(y)$ is linear in its arguments.
Hence, describing the transformation for only a set of basis functions is enough!
\[
    f(x) := e^{i\omega x} = \sum_{n=0}^\infty \frac{{(i\omega)}^n}{n!} x^n
\]
\[
    \mapsto g(y) = \sum_{n=0}^\infty {(i\omega y)}^n = \inv{1 - i\omega y}
\]
\begin{align*}
    \therefore g(y)
    &= \int \frac{d\omega}{\sqrt{2\pi}} \Tilde{f}(\omega) \cdot \inv{1 - i\omega y} \\
    &= \int \frac{d\omega}{\sqrt{2\pi}} \inv{1 - i\omega y}
     \cdot \int \frac{dx}{\sqrt{2\pi}} f(x) e^{-i\omega x} \\
    &= \inv{2\pi} \int d\omega \int dx f(x) \frac{e^{-i\omega x}}{1 - i\omega y}
\end{align*}
Here, the ranges of the integrals are assumed to be over the entire real axis.

