%%%%%%%%%%%%%%%%%%%%%%%
\item
%%%%%%%%%%%%%%%%%%%%%%%

\begin{align*}
    \Tilde{\psi}(\mathbb{k})
    &= \int \frac{d^3 x}{{(2\pi)}^{3/2}} \psi(\mathbb{x}) e^{-i\mathbb{k} \cdot \mathbb{x}} \\
    &= \inv{16\pi^2 a_0^{5/2}} \int d^3 x z e^{-r/2a_0 - i\mathbb{k} \cdot \mathbb{x}} \\
    &= \frac{i}{16\pi^2 a_0^{5/2}} \partialderiv{k_z}
       \int d^3 x e^{-r/2a_0 - i\mathbb{k} \cdot \mathbb{x}} \\
    &= \frac{i}{16\pi^2 a_0^{5/2}} \partialderiv{k_z}
       \int_0^\infty dr r^2 \int_{-1}^1 d(\cos\theta) \int_0^{2\pi} d\phi e^{-r/2a_0 - ikr\cos\theta} \\
    &= \frac{i}{8\pi a_0^{5/2}} \partialderiv{k_z}
       \int_0^\infty dr r^2 e^{-r/2a_0} \diffBetween{-\frac{i}{kr}e^{-ikr\cos\theta}}{\theta=0}{\theta=\pi} \\
    &= \inv{8\pi a_0^{5/2}} \partialderiv{k_z}
       \left( \inv{k} \int_0^\infty dr r e^{-r/2a_0} \left( e^{ikr} - e^{-ikr} \right) \right) \\
    &= \frac{i}{8\pi a_0^{5/2}} \frac{\partial k}{\partial k_z} \partialderiv{k}
       {\left( k^2 + \inv{4a_0^2} \right)}^{-2} \\
    &= \inv{2i\pi a_0^{3/2}} \frac{k_z}{{\left( k^2 + 1/4a_0^2 \right)}^3}
\end{align*}
(Note: The original integral is not spherically symmetric,
since the integrand is not a coordinate-independent scalar (it depends on $z$).
Consequently, the resulting expression depends on more than just the magnitude of $\mathbb{k}$.
However, on the fourth equality sign, we arbitrarily chose the $z$ axis to be along $\mathbb{k}$.
This is because upon applying Feynman's trick, the integrand becomes spherically symmetric again.
This exercise is a good training for applying symmetry arguments only when applicable.)