%%%%%%%%%%%%%%%%%%%%%%%
1.
%%%%%%%%%%%%%%%%%%%%%%%
As $f(x)$ is an even function, we immediately see that no sine terms will be present in the series.
Let
\[
    f(x) = \frac{A_0}{2} + \sum_{n=1}^\infty A_n \cos \frac{2n\pi}{L}x.
\]
\[
    A_0
    = \frac{2}{L} \int_{-\frac{L}{2}}^{\frac{L}{2}} dx f(x)
    = \frac{4}{L} \int_0^{\frac{L}{2}} dx \left( 1 - \frac{2}{L} x \right)
    = 1
\]
\begin{align*}
    A_n
    &= \frac{2}{L} \int_{-\frac{L}{2}}^{\frac{L}{2}} dx f(x) \cos \frac{2n\pi}{L}x \\
    &= \frac{4}{L} \int_0^{\frac{L}{2}} dx \left( 1 - \frac{2}{L} x \right) \cos \frac{2n\pi}{L}x \\
    &= \frac{2\left( 1 - {(-1)}^n \right)}{\pi^2 n^2} \\
    &= \begin{cases}
        \frac{4}{\pi^2 n^2} & (2 \nmid n) \\
        0                   & (2 \mid  n) \\
    \end{cases}
    \quad (n \neq 0)
\end{align*}
\begin{align*}
    \therefore f(x)
    &= \half + \frac{4}{\pi^2} \sum_{n=0}^\infty \inv{{(2n+1)}^2} \cos \frac{2 (2n+1) \pi}{L} x \\
    &= \half + \frac{4}{\pi^2} \left(
        \cos \frac{2\pi}{L} x
        + \inv{9} \cos \frac{6\pi}{L} x
        + \inv{25} \cos \frac{10\pi}{L} x
        + \cdots
    \right)
\end{align*}

