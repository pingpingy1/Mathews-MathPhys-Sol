%%%%%%%%%%%%%%%%%%%%%%%
14.
%%%%%%%%%%%%%%%%%%%%%%%
In matrix form, the coupled differential equations can be written as
\[
    \frac{d}{dt} \begin{pmatrix} N_1 \\ N_2 \\ N_3 \end{pmatrix}
    = \begin{pmatrix}
        -\lambda_1 &            &            \\
         \lambda_1 & -\lambda_2 &            \\
                   &  \lambda_2 & -\lambda_3 \\
    \end{pmatrix}
    \begin{pmatrix} N_1 \\ N_2 \\ N_3 \end{pmatrix}.
\]
Let
\[
    F_i := \Lapltransf{N_i}(s).
\]
\[
    \Rightarrow \Lapltransf{\dot{N_i}}(s)
    = sF_i - N_i(0)
\]
\[
    \Rightarrow \begin{pmatrix} sF_1 - N \\ sF_2 \\ sF_3 - n \end{pmatrix}
    = \begin{pmatrix}
        -\lambda_1 &            &            \\
         \lambda_1 & -\lambda_2 &            \\
                   &  \lambda_2 & -\lambda_3 \\
    \end{pmatrix}
    \begin{pmatrix} F_1 \\ F_2 \\ F_3 \end{pmatrix}
\]
\[
    \Rightarrow \begin{pmatrix}
        s + \lambda_1 &               &               \\
           -\lambda_1 & s + \lambda_2 &               \\
                      &    -\lambda_2 & s + \lambda_3 \\
    \end{pmatrix}
    \begin{pmatrix} F_1 \\ F_2 \\ F_3 \end{pmatrix}
    = \begin{pmatrix} N \\ \\ m \end{pmatrix}
\]
\begin{align*}
    \Rightarrow F_3
    &= \frac{n}{s + \lambda_3} - \frac{N\lambda_1 \lambda_2}{(\lambda_1 - \lambda_2) (\lambda_2 - \lambda_3) (\lambda_3 - \lambda_1)} \\
    &\quad\cdot \left(
        \frac{\lambda_2 - \lambda_3}{s - \lambda_1} +
        \frac{\lambda_3 - \lambda_1}{s - \lambda_2} +
        \frac{\lambda_1 - \lambda_2}{s - \lambda_3}
    \right)
\end{align*}
\begin{align*}
    \therefore N_3(t)
    &= ne^{-\lambda_3 t} - \frac{N\lambda_1 \lambda_2}{(\lambda_1 - \lambda_2) (\lambda_2 - \lambda_3) (\lambda_3 - \lambda_1)} \\
    &\quad\cdot \left(
        (\lambda_2 - \lambda_3) e^{-\lambda_1} +
        (\lambda_3 - \lambda_1) e^{-\lambda_2} +
        (\lambda_1 - \lambda_2) e^{-\lambda_3}
    \right)
\end{align*}

(Note 1: What if two of the $\lambda_i$'s are the same?
In this case, the inverse Laplace transform becomes more complicated,
and the solution would take the form of
either $te^{-t}$ (multiplicity 2) or $t^2 e^{-t}$ (multiplicity 3).)

(Note 2: Solving the linear system of equations, in this case is straightforward
since the matrix is already lower triangular.
However, since only $F_3$ is of interest in our case, the
\urlfoot{https://en.wikipedia.org/wiki/Cramer\%27s_rule}{Cramer's rule}
may be more efficient, especially because the determinant calculations are also relatively simple.)

