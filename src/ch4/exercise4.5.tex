%%%%%%%%%%%%%%%%%%%%%%%
5.
%%%%%%%%%%%%%%%%%%%%%%%
Let
\[
    f(\theta) := \sum_{n=0}^\infty \frac{\cos (2n+1) \theta}{{(2n+1)}^2}.
\]
\[
    \Rightarrow f''(\theta)
    = -\sum_{n=0}^\infty \cos (2n+1) \theta
    = -\Re\left\{ \sum_{n=0}^\infty e^{i(2n+1)\theta} \right\}
    = 0
\]
\[
    \Rightarrow f'(\theta)
    = f'\left( \halfpi \right)
    = -\sum_{n=0}^\infty \frac{{(-1)}^n}{2n+1}
    = -\frac{\pi}{4}
\]
\[
    \Rightarrow f(\theta)
    = f(0) - \frac{\pi}{4} \theta
    = \sum_{n=0}^\infty \inv{{(2n+1)}^2} - \frac{\pi}{4} \theta
    = \frac{\pi^2}{8} \left( 1 - \frac{2}{\pi}\theta \right)
\]

Now note that $f$ must be periodic and even, which this linear function definitely is not.
Yet, nowhere in our derivation did we specify the range of $\theta$!
What went wrong?

The answer is that our $f''(\theta)$ is not strictly true;
it diverges at integral multiples of $\pi$.
Hence, we cannot invoke the fundamental theorem of calculus to obtain $f'(\theta)$ in one go.
However, the above derivation works on the interval $(0,\pi)$,
which we periodically and evenly extend to the entire real line.

\[
    \therefore f(\theta) = \begin{cases}
        \frac{\pi^2}{8} \left( 1 - \frac{2}{\pi} (\theta - 2k\pi) \right) & (2k\pi < \theta < (2k+1)\pi) \\
        \frac{\pi^2}{8} \left( 1 + \frac{2}{\pi} (\theta - 2k\pi) \right) & ((2k-1)\pi < \theta < 2k\pi) \\
    \end{cases}
\]

(Note: What about the part where we evaluate $f(0)$? Is that also erroneous?
Technically yes, but the diligent mathematicians have proven
\urlfoot{https://en.wikipedia.org/wiki/Abel\%27s_theorem}{Abel's theorem},
with which we may correctly write
\[
    \lim_{\theta \rightarrow 0+} f(\theta) = \frac{\pi^2}{8}.
\]
Of course, we are physicists, so \emph{obviously} it works!)

