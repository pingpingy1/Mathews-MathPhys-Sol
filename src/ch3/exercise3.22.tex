%%%%%%%%%%%%%%%%%%%%%%%
\item
%%%%%%%%%%%%%%%%%%%%%%%

We shall perform a contour integral of $f(z) := \frac{\sin z}{z \left( a^2 + z^2 \right)}$ along \reffig{}~\ref{fig:problem3-12}.
Note that the singularity at the origin is removable, as we may define
$f(0) := \lim_{z\rightarrow 0} f(z) = \inv{a^2}$.
\[
	\oint_C dz f(z)
	= 2i\pi \res{z = ia} f(z)
	= 2i\pi \cdot \evalAt{\frac{\sin z}{z \left( z + ia \right)}}{z = ia}
	= \frac{\pi \sin a}{a^2}
\]
We also have
\begin{align*}
	\oint_C dx f(z)
    &= \int_{-R}^{R} dx \frac{\sin x}{x \left( a^2 + x^2 \right)}
     + \int_0^{\pi} d\theta iRe^{i\theta} \frac{\sin\left( Re^{i\theta} \right)}{Re^{i\theta} \left( a^2 + R^2 e^{2i\theta} \right)}     \\
    &\xrightarrow{R \rightarrow \infty} 2\int_0^\infty dx \frac{\sin x}{x \left( a^2 + x^2 \right)}.
\end{align*}
(Note: One may object that it is nontrivial that the second integral vanishes in the limit,
as $\sin z$ grows without bound along the imaginary axis.
A physicist like myself would sweep this fact under the rug,
but if any mathematician would like to give a more rigorous proof of this fact, I'm all ears!)
\[
	\therefore \int_0^\infty dx \frac{x^2}{{a^2 + x^2}^3} = \frac{\pi \sin a}{2a^2}
\]