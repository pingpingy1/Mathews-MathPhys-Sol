%%%%%%%%%%%%%%%%%%%%%%%
\item
%%%%%%%%%%%%%%%%%%%%%%%

\begin{figure}[ht]
	\centering
	\begin{tikzpicture}
		% Axes
		\draw[axis] (-2cm, 0cm) -- (2cm, 0cm) node[right] {$\Re$};
		\draw[axis] (0cm, -1cm) -- (0cm, 3cm) node[right] {$\Im$};

		% Poles
        \node[pole] (A) at (0cm, -0.7cm) {}; \node[right] at (A.east) {$-i$};
        \node[pole] (B) at (0cm,  0.7cm) {}; \node[right] at (B.east) {$i$};
        \node[pole] (C) at (0cm,  1.4cm) {}; \node[right] at (C.east) {$2i$};
        \node[pole] (D) at (0cm,  2.1cm) {}; \node[right] at (D.east) {$3i$};

		% Contour
		\def\R{1.8cm}
        \draw[contour] (-\R, 0cm) node[below] {$-R$} -- (\R, 0cm) node[below] {$R$};
		\draw[contour] (\R, 0cm) arc (0:180:\R);
        \node[blue, left, yshift=0.2cm] at (0cm, \R) {$iR$};
	\end{tikzpicture}
	\caption{Contour for \refprob{}~3--17.}%
	\label{fig:problem3-17}
\end{figure}

We must first determine whether, or more precisely, for which values of $a$, this integral converges.
As $\abs{x} \rightarrow \infty$, $\sinh ax \sim e^{\abs{ax}}$.
Thus, this integral converges whenever $\abs{a} < \pi$.

Let $f(z) := \frac{\sinh az}{\sinh \pi z}$
The singularity at the origin is removable, as
\[
    \lim_{z \rightarrow 0} \frac{\sinh az}{\sinh \pi z}
    = \lim_{z \rightarrow 0} \frac{a \cosh az}{\pi \cosh \pi z}
    = \frac{a}{\pi}
\]
via L'H\^opital's rule.
(Note the sloppy math here:
the first equality hold only because the second expression does indeed converge to a finite number.
If this were a calculus class, this equation is prone to points lost;
since this is physics, we can proceed as is.)
Thus, $f(z)$ has poles at $z = im,\; m \in \mathbb{Z} \backslash \{0\}$.
We ignore cases where $a$ is a rational multiple of $\pi$ for convinience;
these cases are left as exercise for the reader.\footnote{
Just kidding!
I will come back for this later, but if you'd like to help,
please consider discussing on \url{https://github.com/pingpingy1/Mathews-MathPhys-Sol/issues/1}.
}

Consider the contour shown in \reffig~\ref{fig:problem3-17}.
\begin{align*}
    \oint_C dz f(z)
    &\xrightarrow{R \rightarrow \infty} 2i\pi \sum_{m=1}^\infty \res{z = im} f(z) \\
    &= 2i \sum_{m=1}^\infty \frac{\sinh iam}{\cosh i\pi m} \\
    &= 2\sum_{m=1}^\infty {(-1)}^{m-1} \sin am \\
    &= 2 \Im\left\{ \sum_{m=1}^\infty e^{i(m-1)\pi} e^{iam} \right\} \\
    &= \tan \frac{a}{2}
\end{align*}
We also have
\begin{align*}
    \oint_C dz f(z)
    &= \int_{-R}^R dx \frac{\sinh ax}{\sinh \pi x}
     + \int_0^\pi d\theta iRe^{i\theta} \frac{\sinh\left( aRe^{i\theta} \right)}{\sinh\left( \pi Re^{i\theta} \right)} \\ 
    &= \int_{-R}^R dx \frac{\sinh ax}{\sinh \pi x}
     + \bigO\left( \frac{R}{e^{\left( \pi - \abs{a} \right)}R} \right) \\
    &\xrightarrow{R \rightarrow \infty} \int_{-\infty}^\infty dx \frac{\sinh ax}{\sinh \pi x}
\end{align*}

\[
    \therefore \int_{-\infty}^\infty dx \frac{\sinh ax}{\sinh \pi x} = \tan \frac{a}{2}
\]
