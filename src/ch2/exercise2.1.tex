%%%%%%%%%%%%%%%%%%%%%%%
\item
%%%%%%%%%%%%%%%%%%%%%%%

A useful trick when dealing with complex sign patterns like this is to represent them using an exponential.
In this case, we acknowledge that
\[
	\left\{ \Re \left\{ e^{i\left( \frac{n}{2} - \frac{1}{4} \right) \pi} \right\} \right\}
	= \frac{1}{\sqrt{2}},\; \frac{1}{\sqrt{2}},\; -\frac{1}{\sqrt{2}},\; -\frac{1}{\sqrt{2}}, \cdots
\]
and write the series as
\begin{align*}
	(\text{Given series})
	&= \sum_{n=0}^\infty \frac{1}{4^n} \cdot \sqrt{2} \Re \left\{ e^{i\left( \frac{n}{2} - \frac{1}{4} \right) \pi} \right\} \\
	&= \Re \left\{ \sqrt{2} e^{-i\frac{\pi}{4}} \sum_{n=0}^\infty {\left( \frac{e^{i\frac{\pi}{4}}}{2} \right)}^n \right\} \\
	&= \Re \left\{ (1 - i) \sum_{n=0}^\infty {\left( \frac{i}{4} \right)}^n \right\} \\
	&= \Re \left\{ (1 - i) \cdot \frac{1}{1 - \frac{i}{4}} \right\} \\
	&= \frac{16}{17} \Re \left\{ (1 - i) \left( 1 + \frac{i}{4} \right) \right\} \\
	&= \frac{16}{17} \left( 1 + \frac{1}{4} \right) \\
	&= \frac{20}{17}.
\end{align*}
For series problems, it is always a good idea to verify the results numerically with a calculator, provided that the series converges quickly enough.
In this case, evaluating upto $\frac{1}{1024}$ yields about $1.1768$ while the correct answer is about $1.1765$.