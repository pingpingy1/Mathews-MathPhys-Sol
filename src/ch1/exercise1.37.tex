%%%%%%%%%%%%%%%%%%%%%%%
37.
%%%%%%%%%%%%%%%%%%%%%%
We first take note that $y = x + \alpha$ is a trivial solution to the equation, as is easily verifiable.
Let $z := y - x$.
We directly have
\[
	z'' = z^2 - e^{2z}.
\]
We multiply each side by $z'$ and integrate each side to obtain
\[
	\frac{{z'}^2}{2} = \frac{1}{3}z^3 - \half e^{2z} + E.
\]
I find it illuminating to consider $z$ as the position of a particle of mass $1$.
The above equation could then be interpreted as the energy conservation of this particle as it moves under the influence of the potential
\[
	V(z) = -\frac{1}{3}x^3 + \frac{1}{2}e^{2z}
\]
and retains its total energy $E$.
Thus, we could consider the infinitesimal oscillation about $z=\alpha$ as a harmonic oscillator.
\[
	\left. \frac{d^2 V}{dz^2} \right|_{z = \alpha}
	= -2\alpha + 2e^{2\alpha} = 2\alpha (\alpha - 1)
	= m\omega^2 = \omega^2
\]
\[
	\Rightarrow \omega = \sqrt{2\alpha(\alpha - 1)}
\]
\[
	\Rightarrow z \approx C_1 \sin \left(
	\sqrt{2\alpha (\alpha - 1) x} + C_2
	\right)
\]
\[
	\therefore y \approx x + C_1 \sin \left(
	\sqrt{2\alpha (\alpha - 1) x} + C_2
	\right)
\]

(Note: More accurate expressions may be obtained using the higher-order terms of the potential energy and perturbation methods.)

