%%%%%%%%%%%%%%%%%%%%%%%
\item
%%%%%%%%%%%%%%%%%%%%%%
\begin{enumerate}[wide, labelindent=0pt, label= (\alph*)]
\item \mbox{}\\
Assume $y = \sum_{n=0}^\infty c_n x^n$.
Using
\[
	y'' = y \left( x^2 - y^2 \right),\; x^2 = \sum_{n=0}^\infty \delta_{n2} x^n,
\]
and
\[
	y^2 = \sum_{n=0}^\infty \left( \sum_{k=0}^n c_k c_{n - k} \right) x^n,
\]
one can find (after algebraic tedium) that
\[
	y \left( x^2 - y^2 \right)
	= -c_0^2 - 3c_1 c_0^2 x + \sum_{n=2}^\infty \left(
		c_{n - 2} - \sum_{k=0}^n \sum_{l=0}^k c_l c_{k - l} c_{n - k}
	\right) x^n
\]
\begin{align*}
	\Rightarrow
	&\left( 2c_2 + c_0^2 \right)
	+ \left( 6c_3 + 3c_1 c_0^2 \right) x \\
	&+ \sum_{n=2}^\infty \left(
		(n + 2)(n + 1) c_{n + 2} - c_{n - 2}
		+ \sum_{k=0}^n \sum_{l=0}^k c_l c_{k - l} c_{n - k}
	\right) x^n = 0.
\end{align*}

We thus find two degrees of freedom, namely $c_0 = A$ and $c_1 = B$.
\[
	c_2 = -\frac{A^2}{2},\;
	c_3 = -\frac{A^2 B}{2},\;
	c_{n + 2} = -\frac{1}{(n + 2)(n + 1)} \left(
		c_{n - 2} - \sum_{k=0}^n \sum_{l = 0}^k c_l c_{k - l} c_{n - k}
	\right)
\]
We could calculate a few terms using the recurrence relation:
\[
	c_4
	= \frac{1}{4 \cdot 3} \left(
		c_0 - 3c_2 c_0^2 - 3c_1^2 c_0
	\right)
	= \frac{A}{12} \left(
		1 + \frac{3}{2}A^3 - 3B^2
	\right),
\]
\begin{align*}
	c_5
	&= \frac{1}{5 \cdot 4} \left(
		c_1 - 3c_3 c_0^2 - 6c_2 c_1 c_0 - 3c_2 c_1
	\right) \\
	&= \frac{B}{20} \left(
		1 + 3A^3 \left( \frac{A}{4} + 1 \right)
	\right)
\end{align*}
\begin{align*}
	\therefore
	y
	=& A + Bx
	- \frac{A^2}{2}x^2
	- \frac{A^2 B}{2}x^3 \\
	&+ \frac{A}{12} \left(
		1 + \frac{3}{2}A^3 - 3B^2
	\right) x^4
	+ \frac{B}{20} \left(
		1 + 3A^3 \left( \frac{A}{4} + 1 \right)
	\right) x^5
	+ \cdots
\end{align*}

\item \mbox{}\\
We iteratively approximate the particular nonoscillating solution, for which the second derivative would stay relatively small.
That is, we use the following iterative scheme:
\[
	-xy^{(0)} + {y^{(0)}}^3 = 0,\;
	y^{(n + 1)} = \sqrt[3]{xy^{(n)} - {y^{(n)}}''}
\]
We thus get
\[
	y^{(0)} = \sqrt{x} = x^{1/2}
\]
\[
	y^{(1)}
	= \sqrt[3]{x^{3/2} + \frac{1}{4x^{3/2}}}
	\approx x^{1/2} + \frac{1}{12} x^{-5/2}
\]
\begin{align*}
	y^{(2)}
	&= \sqrt[3]{x^{3/2} + \frac{1}{3}x^{-3/2} - \frac{35}{48}x^{-9/2}} \\
	&\approx x^{1/2} + \frac{1}{9}x^{-5/2} - \frac{35}{144}x^{-11/2}
\end{align*}
\begin{align*}
	y^{(3)}
	&= \sqrt[3]{x^{3/2} + \frac{13}{36}x^{-3/2} - \frac{175}{144}x^{-9/2} + \frac{5005}{576}x^{-15/2}} \\
	&\approx x^{1/2} + \frac{13}{108}x^{-5/2} - \frac{175}{432}x^{-11/2} + \frac{5005}{1728}x^{-17/2}
\end{align*}
\[
	\therefore y \approx x^{1/2} + \frac{13}{108}x^{-5/2} - \frac{175}{432}x^{-11/2} + \frac{5005}{1728}x^{-17/2}
\]

(Note: One may find more accurate approximations by (i) iterating more times, or (ii) expanding to more terms for each binomial expansion above.)

\end{enumerate}