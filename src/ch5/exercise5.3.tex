%%%%%%%%%%%%%%%%%%%%%%%
3.
%%%%%%%%%%%%%%%%%%%%%%%
\begin{figure}[ht]
    \centering
    \begin{subfigure}{0.4\textwidth}
        \centering
        \begin{tikzpicture}
            % Axes
            \draw[axis] (-2cm, 0cm) -- (2cm, 0cm) node[right] {$\Re$};
            \draw[axis] (0cm, -2cm) -- (0cm, 2cm) node[right] {$\Im$};
            \node (name) at (1.5cm, 1.5cm) {$z$};
            \draw (name.north west) |- (name.south east);
    
            % Charges & Conductors
            \node[pole] (A) at (0cm, 0.8cm) {}; \node[right, red] at (A.east) {$+\lambda$};
                                                \node[left]       at (A.west) {$a$};
            \draw[charged strip] (-1.8cm, 0cm) -- (1.8cm, 0cm);
            \draw[charged strip, blue] (-1.8cm, 1.2cm) -- (1.8cm, 1.2cm);
            \node[right] at (0cm, 1.4cm) {$b$};

            % Ground connections
            \draw[orange] (1.5cm, 0cm) -- (1.5cm, -0.2cm);
            \draw[orange] (1.2cm, -0.2cm) -- (1.8cm, -0.2cm);
            \draw[orange] (1.3cm, -0.3cm) -- (1.7cm, -0.3cm);
            \draw[orange] (1.4cm, -0.4cm) -- (1.6cm, -0.4cm);

            \draw[blue] (1.5cm, 1.2cm) -- (1.5cm, 1.0cm);
            \draw[blue] (1.2cm, 1.0cm) -- (1.8cm, 1.0cm);
            \draw[blue] (1.3cm, 0.9cm) -- (1.7cm, 0.9cm);
            \draw[blue] (1.4cm, 0.8cm) -- (1.6cm, 0.8cm);
            
        \end{tikzpicture}
        \caption{}%
        \label{subfig:problem5-3:original}
    \end{subfigure}
    \hfill
    \begin{subfigure}{0.4\textwidth}
        \centering
        \begin{tikzpicture}
            % Axes
            \draw[axis] (-2cm, 0cm) -- (2cm, 0cm) node[right] {$\Re$};
            \draw[axis] (0cm, -2cm) -- (0cm, 2cm) node[right] {$\Im$};
            \node (name) at (1.5cm, 1.5cm) {$\zeta$};
            \draw (name.north west) |- (name.south east);
    
            % Charges & Conductors
            \draw[dotted, gray] (0cm, 0cm) -- (-1cm, 1.732cm)
                node[above] {$\theta = \frac{a}{b}\pi$};
            \draw[dotted, gray] (1cm, 0cm) arc (0:360:1cm);
            \node[right, gray] at (0.7cm, 0.7cm) {$r = 1$};
            \node[pole] (A) at (-0.5cm, 0.866cm) {}; \node[left, red] at (A.west) {$+\lambda$};

            \draw[charged strip] (0cm, 0cm) -- (1.8cm, 0cm);
            \draw[charged strip, blue] (-1.8cm, 0cm) -- (0cm, 0cm);

            \draw[dotted, gray] (0cm, 0cm) -- (-1cm, -1.732cm)
                node[above] {$\theta = -\frac{a}{b}\pi$};
            \node[right, gray] at (0.7cm, 0.7cm) {$r = 1$};
            \node[pole, opacity = 0.5] (B) at (-0.5cm, -0.866cm) {};
                \node[left, red, opacity=0.5] at (B.west) {$-\lambda$};

            % Ground connections
            \draw[orange] (1.5cm, 0cm) -- (1.5cm, -0.2cm);
            \draw[orange] (1.2cm, -0.2cm) -- (1.8cm, -0.2cm);
            \draw[orange] (1.3cm, -0.3cm) -- (1.7cm, -0.3cm);
            \draw[orange] (1.4cm, -0.4cm) -- (1.6cm, -0.4cm);

            \draw[blue] (-1.5cm, 0cm) -- (-1.5cm, -0.2cm);
            \draw[blue] (-1.2cm, -0.2cm) -- (-1.8cm, -0.2cm);
            \draw[blue] (-1.3cm, -0.3cm) -- (-1.7cm, -0.3cm);
            \draw[blue] (-1.4cm, -0.4cm) -- (-1.6cm, -0.4cm);
            
        \end{tikzpicture}
        \caption{}%
        \label{subfig:problem5-3:final}
    \end{subfigure}
    \caption{Results of conformal mappings used in \refprob~5--3.
    Note that the grounded symbols have not been mapped conformally.}%
    \label{fig:problem5-3}
\end{figure}

The effects of the given transformation is shown in \reffig~\ref{fig:problem5-3}.
The region between the plates in the $z$ plane
is mapped to the upper half plane in the $\zeta$ plane.
Moreover, the two charged plates are mapped onto the positive and negative halves of the real axis.
Therefore, we should find a complex function $F(\zeta)$
such that its imaginary part vanishes on the real axis
while showing a log-like behavior near $e^{i\frac{a}{b}\pi}$.
Those with some electrostatic problem-solving experience may recall the
\urlfoot{https://en.wikipedia.org/wiki/Method_of_image_charges}{image charge method}.
Those without could tinker around, perhaps placing some negative charges trying to
cancel out the potential on the real axis.
This may lead you to placing a charged line of charge density $-\lambda$
at $\zeta = e^{-i\frac{a}{b}\pi}$.

\[
    \Rightarrow F(\zeta)
    = \frac{i\lambda}{2\pi\epsilon_0} \left(
        \ln \left( \zeta - e^{-i\frac{a}{b}\pi} \right)
        - \ln \left( \zeta - e^{i\frac{a}{b}\pi} \right)
    \right)
\]
\begin{align*}
    \Rightarrow f(z)
    &= F(\zeta(z)) \\
    &= \frac{i\lambda}{2\pi\epsilon_0} \left(
        \ln \left( e^{i\frac{z}{b}} - e^{-i\frac{a}{b}\pi} \right)
        - \ln \left( e^{i\frac{z}{b}} - e^{i\frac{a}{b}\pi} \right)
    \right)
\end{align*}
\begin{align*}
    \therefore V(x, y)
    &= \Im \left\{ f(z) \right\} \\
    &= \frac{\lambda}{2\pi\epsilon_0} \Re \left\{ \left(
        \ln \left( e^{i\frac{x + iy}{b}} - e^{-i\frac{a}{b}\pi} \right)
        - \ln \left( e^{i\frac{x + iy}{b}} - e^{i\frac{a}{b}\pi} \right)
    \right) \right\}
\end{align*}

