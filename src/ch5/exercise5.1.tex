%%%%%%%%%%%%%%%%%%%%%%%
\item
%%%%%%%%%%%%%%%%%%%%%%%

\begin{figure}[ht]
    \centering
    \begin{subfigure}{\textwidth}
        \centering
        \begin{tikzpicture}
            % Axes
            \draw[axis] (-3.7cm, 0cm) -- (3.7cm, 0cm) node[right] {$\Re$};
            \draw[axis] (0cm, -2cm) -- (0cm, 2cm) node[right] {$\Im$};
            \node (name) at (3.2cm, 1.5cm) {$z$};
            \draw (name.north west) |- (name.south east);
    
            % Charges
            \draw[charged strip]
                (-0.8cm, 0cm) -- node[midway, above] {$+\lambda$} (0.8cm, 0cm);
            \draw[charged strip]
                (1.4cm, 0cm) -- node[midway, above] {$+\lambda$} (3cm, 0cm);
            \draw[charged strip]
                (-3cm, 0cm) -- node[midway, above] {$+\lambda$} (-1.4cm, 0cm);
            \node[below] at (   0cm, 0cm) {$2a$};
            \node[below] at ( 2.2cm, 0cm) {$2a$};
            \node[below] at (-2.2cm, 0cm) {$2a$};
            \node[above] at ( 1.1cm, 0cm) {$2b$};
            \node[above] at (-1.1cm, 0cm) {$2b$};

            \draw[dashed, gray] ( 1.1cm, -1.8cm) -- ( 1.1cm, 1.8cm);
            \draw[dashed, gray] (-1.1cm, -1.8cm) -- (-1.1cm, 1.8cm);

        \end{tikzpicture}
        \caption{}%
        \label{subfig:problem5-1:original}
    \end{subfigure}
    \hfill
    \begin{subfigure}{0.4\textwidth}
        \centering
        \begin{tikzpicture}
            % Axes
            \draw[axis] (-2cm, 0cm) -- (2cm, 0cm) node[right] {$\Re$};
            \draw[axis] (0cm, -2cm) -- (0cm, 2cm) node[right] {$\Im$};
            \node (name) at (1.5cm, 1.5cm) {$\zeta$};
            \draw (name.north west) |- (name.south east);
    
            % Charges
            \draw[charged strip] (1cm, 0cm) arc (0:131:1cm);
            \draw[charged strip] (1cm, 0cm) arc (0:-131:1cm);
            \node[right, yshift=0.2cm, orange] at (0.65cm, 0.65cm) {$+\lambda$};
            \node[below, xshift=0.2cm] at (1cm, 0cm) {$1$};
            \draw[dashed, gray] (0cm, 0cm)
                -- (-1.312cm,  1.509cm) node[above] {$\theta =  \frac{\pi a}{a + b}$};
            \draw[dashed, gray] (0cm, 0cm)
                -- (-1.312cm, -1.509cm) node[below] {$\theta = -\frac{\pi a}{a + b}$};
            
        \end{tikzpicture}
        \caption{}%
        \label{subfig:problem5-1:exp}
    \end{subfigure}
    \hfill
    \begin{subfigure}{0.4\textwidth}
        \centering
        \begin{tikzpicture}
            % Axes
            \draw[axis] (-2cm, 0cm) -- (2cm, 0cm) node[right] {$\Re$};
            \draw[axis] (0cm, -2cm) -- (0cm, 2cm) node[right] {$\Im$};
            \node (name) at (1.5cm, 1.5cm) {$\eta$};
            \draw (name.north west) |- (name.south east);
    
            % Charges
            \draw[charged strip] (-1cm, 0cm) -- node[midway, above] {$+\lambda$} (1cm, 0cm);
            \node[below] at ( 1cm, 0cm) {$ \tan \frac{\pi a}{2(a+b)}$};
            \node[below] at (-1cm, 0cm) {$-\tan \frac{\pi a}{2(a+b)}$};
            
        \end{tikzpicture}
        \caption{}%
        \label{subfig:problem5-1:final}
    \end{subfigure}
    \caption{Results of conformal mappings used in \refprob~5--1.}%
    \label{fig:problem5-1}
\end{figure}

The problem describes a linear array of charged stripts,
as shown in \reffig~\ref{subfig:problem5-1:original}.
We wish to find a conformal map that transforms this into a problem with a known solution.
It is helpful to approach this step by step.

The first problem we see is that there are infinitely many charged sheets;
we wish to collapse them into a single strip.
To do so, we should employ a map that is periodic with period $2(a+b)$.
Hence, we employ the map
\[
    \zeta := \exp\left( i\frac{\pi z}{a + b} \right)
    \Leftrightarrow z = -i (a + b) \ln\zeta.
\]
This yields \reffig~\ref{subfig:problem5-1:exp}, where the charged strips collapse into
a single one shaped as a circular arc (or a cylindrical arc in 3D).

We now want to map this circular arc into a well-known shape.
One way to do so is given by the
\urlfoot{https://en.wikipedia.org/wiki/M\%C3\%B6bius_transformation}{M\"obius transformation},
which is often introduced as an important class of conformal mappings in other textbooks.
With that motivation-via-divine-intervention aside, let us consider the map
\[
    \eta = \frac{a\zeta + b}{c\zeta + d}.
\]
We will attempt to transform the unit circle in the $\zeta$ plane
into the real axis in the $\eta$ plane.
Also, it would be nice to have $1$ map to the origin for more symmetry
and $-1$ to infinity because it is not part of the charged strip.
These conditions constrain the transformation to
\[
    \eta = a \frac{1 - \zeta}{1 + \zeta}.
\]
Finally, requiring that $i$ maps to a point on the real axis, we get
\[
    \eta := i \frac{1 - \zeta}{1 + \zeta}
    \Leftrightarrow \zeta = \frac{i - \eta}{i + \eta}.
\]
This transformation actually maps the point $e^{i\alpha}$ to $\tan\frac{\alpha}{2}$,
so it places the charged strip on the real interval
from $-\tan\frac{\pi a}{2(a+b)}$ to $\tan\frac{\pi a}{2(a+b)}$.
This is shown in \reffig~\ref{subfig:problem5-1:final}.

This final situation in the $\eta$ was dealt with in the main text
and is solved by
\[
    f(\eta) = -\frac{\lambda}{2\pi\epsilon_0} \sin^{-1} \left(
        \frac{\eta}{\tan (\pi a / 2(a + b))}
    \right).
\]
Composing the two conformal maps yields
\[
    \eta(z) = \tan \left( \frac{\pi z}{2(a + b)} \right).
\]
\begin{align*}
    \therefore F(z)
    &= f(\eta(z)) \\
    &= -\frac{\lambda}{2\pi\epsilon_0} \sin^{-1} \left(
        \frac{\tan \left( \frac{\pi z}{2(a + b)} \right)}{\tan \left( \frac{\pi a}{2(a + b)} \right)}
    \right)
\end{align*}
The potential is given by the imaginary part of this function.

(Note: It may be possible to directly motivate the conformal map $\eta(z)$ as follows:
The map should be periodic as discussed earlier,
but we also require that real values are mapped to real values.
This would ensure that the charged strips stay straight, which may be easier to calculate.
The tangent function achieves this as shown in this solution,
but one may be tempted to use one of the other trigonometric functions.
The reader is encouraged to try out other possible mappings.)