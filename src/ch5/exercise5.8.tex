%%%%%%%%%%%%%%%%%%%%%%%
8.
%%%%%%%%%%%%%%%%%%%%%%%
Recall that the complex potential is given by
\[
    f(\zeta) = -\frac{\lambda}{2\pi\epsilon_0} \sin^{-1} \frac{\zeta}{a}.
\]
Thus, our problem reduces to expressing approximate forms of the arcsine for large arguments.
To this end, let $w := \sin^{-1} z$.
\[
    z = \sin w = \frac{e^{iw} - e^{-iw}}{2i}
\]
\[
    \Rightarrow e^{2iw} - 2ize^{iw} - 1 = 0
\]
\[
    \Rightarrow e^{iw} = iz \pm \sqrt{1 - z^2}
\]
The arcsine is conventionally defined using the plus sign here
(since we want $w \in \left[ 0, \halfpi \right]$ when $z \in [0, 1]$),
which yields
\[
    w = \sin^{-1} z = -i\ln\left( iz + \sqrt{1 - z^2} \right).
\]
\begin{align*}
    \Rightarrow f(\zeta)
    &= \frac{i\lambda}{2\pi\epsilon_0} \ln\left(
        i\frac{\zeta}{a} + \sqrt{1 - {\left( \frac{\zeta}{a} \right)}^2}
    \right) \\
    &\approx \frac{i\lambda}{2\pi\epsilon_0} \ln\left( 2i\frac{\zeta}{a} \right)
\end{align*}
\begin{align*}
    \therefore V(x, y)
    &= \Im\left\{ f(\zeta) \right\} \\
    &\approx \frac{\lambda}{2\pi\epsilon_0} \Re\left\{ \ln\left( \zeta \right) \right\}
\end{align*}
This is exactly the potential created by a line charge of $\lambda$ at the origin.

